\section{Eigenvalues, Eigenvectors, and
Invariant Subspaces}
\subsection{Invariant Subspaces}
\begin{definition}
    Let $T \in \c L(V,V)$ on a vector space $V \neq \lb 0 \rb$. A subspace $U \subseteq V$ is called an invariant subspace is invariant under $T$ if $Tu \in U \ \forall u \in U$.
\end{definition}
\begin{example}
    For any $T \in \c L(V,V)$, the following subspaces are invariant:
    \begin{enumerate}
        \item $\lb 0 \rb$
        \item $V$
        \item $\nul T = \lb v \in V : Tv = 0 \rb$ \\
        If $Tv \in \nul T$, then $Tv = 0 \in \nul T$.
        \item $\range T = \lb w \in W : w = Tv \text{ for some } v \in V \rb$ \\
        So $Tw \in \range T$.
    \end{enumerate}
\end{example}
\begin{question}
    What are $1$-dimensional invariant subspaces. Then $U = \spa(u)$ for some $u \neq 0$. Invariant means $Tu = \lambda u$ for some $\lambda \in  \b F$, where $u$ is the eigenvector of $T$ and $\lambda$ is the eigenvalues. 
\end{question}
\begin{remark}
    $u \neq 0$ if $u$ is a eigenvector is $T$. $\lambda = 0$ is possible.
\end{remark}
\begin{question}
    How do we find eigenvalues and eigenvalues.
\end{question}
\begin{proposition} Let $T$ be a linear operator, then the following are equivalent
\begin{enumerate}
    \item $\lambda$ is a eigenvalue of $T$.
    \item $T - \lambda\b I$ is not invertible.
    \item $T - \lambda \b I$ is not injective.
    \item $T - \lambda \b I$ is not surjective.
\end{enumerate}
\end{proposition}