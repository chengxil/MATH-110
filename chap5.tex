%!TEX root = ./main.tex
\section{Eigenvalues, Eigenvectors, and
Invariant Subspaces}
\subsection{Invariant Subspaces}
\begin{definition}
    Let $T \in \c L(V,V)$ on a vector space $V \neq \lb 0 \rb$. A subspace $U \subseteq V$ is called an invariant subspace is invariant under $T$ if $T\vec u \in U \ \forall \vec u \in U$.
\end{definition}
\begin{example}
    For any $T \in \c L(V,V)$, the following subspaces are invariant:
    \begin{enumerate}
        \item $\lb 0 \rb$
        \item $V$
        \item $\nul T = \lb \vec v \in V : T\vec v = 0 \rb$ \\
        If $T\vec v \in \nul T$, then $T\vec v = 0 \in \nul T$.
        \item $\range T = \lb \vec w \in W : \vec w = T\vec v \text{ for some } \vec v \in V \rb$ \\
        So $Tw \in \range T$.
    \end{enumerate}
\end{example}
\begin{question}
    What are $1$-dimensional invariant subspaces?  
\end{question}
\begin{answer}
Then $U = \spa(\vec u)$ for some $\vec u \neq 0$. Invariant means $T\vec u = \lambda \vec u$ for some $\lambda \in  \b F$, where $\vec u$ is the eigenvector of $T$ and $\lambda$ is the eigenvalues.
\end{answer}
\begin{remark}
    $\vec u \neq 0$ if $\vec u$ is a eigenvector is $T$. $\lambda = 0$ is possible.
\end{remark}
\begin{proposition} Let $T$ be a linear operator in $V$, then the following are equivalent
\begin{enumerate}
    \item $\lambda$ is a eigenvalue of $T$.
    \item $T - \lambda\b I$ is not invertible.
    \item $T - \lambda \b I$ is not injective.
    \item $T - \lambda \b I$ is not surjective.
\end{enumerate}
\end{proposition}
We have already proven that statement $2,3,4$ are logically equivalent. 
\begin{theorem}
    Suppose $\li vm$ are eigenvectors of $T \in \c L(V)$ corresponding to distinct eigenvalues $\li \lambda m$ will be linearly independent.
\end{theorem}
\begin{proof}
Suppose $\li{\vec v}m$ are linearly independent. By linear dependence lemma, we find a the minimum index $k \leq m$ such that $\vec v_k \in \spa (\li{\vec v}{k-1})$. i.e.
\begin{equation} \label{eqn1}
     \vec v_k = \lincomb{\alpha}{\vec v}{k-1} 
\end{equation}
Apply linear transformation on both sides 
\begin{equation} \label{eqn2}
    T\vec v_k = T\lincomb{\alpha}{\vec v}{k-1} 
\end{equation}
\begin{equation} \label{eqn3}
    \lambda \vec v_k = \alpha_1 \lambda_1 \vec v_1 + \alpha_2 \lambda_2 \vec v_2 + \cdots + \alpha_n \lambda_n \vec v_n 
\end{equation} 
We multiply by equation \ref{eqn1} by $\lambda_m$ and subtract by from \ref{eqn3} and we get \[ 0 = \alpha_1 (\lambda_1 - \lambda_k)\vec v_1 + \alpha_2(\lambda_2 - \lambda_k)\vec v_2 + \cdots + \alpha_{k-1} (\lambda_{k-1} - \lambda_k)\vec v_{k-1}\]
A contradiction since $k$ is not the minimum index with the property chosen above. Therefore the list $\li{\vec v}m$ must be linearly independent.
\end{proof}
\begin{corollary}
An operator $T \in \c L(V)$ has at most $\boxed{\dim V}$ distinct eigenvalues.
\end{corollary}
\subsubsection{Restriction Operators}
\begin{definition}
    Suppose $T \in \c L(V)$ and $U$ is a subspace of $V$ invariant under $T$. Then the restriction operator $T|_U \in \c L(U)$ is deifned by $T|_U(\vec u) = T\vec u$ for all $\vec u \in U$.
\end{definition}
\subsection{Eigenvectors and Upper-Triangular
Matrices}

\subsubsection{Polynomials in T}
\begin{definition}
    Suppose $T \in \c L(V)$, then $T^k$ is defined as
    \[ T^k := \underbrace{k \circ k \circ \cdots \circ k}_{k \text{ times}}\]
    Notice that $T^0 = \b I, T^1 = T$.
\end{definition}
\begin{definition}
    If $p(x) = a_0 + a_1x + \cdots + a_nx^n$, then we can define $p(T)$ as $a_o\b I + a_1T + a_2T + \cdots + a_nT^n$.
\end{definition}
\begin{example}
    Let $V := \c P(\b R), S: p \mapsto 3p'' +  2p' + p, D: p \mapsto p'$. We can see that $S$ can be expressed as $S = D^0 + 2D + 3D^2$. Therefore \[\c M(S) = 3\c M^2(D) + 2\c M(D) + M(\b I)\] we need to have to take the same basis for inputs and output when forming $\c M(\cdot)$.  
    
    \noindent Let's use our favorite basis $1,x,x^2, x^3$. We then can see
    \[ \c M(D) = \bml 0 & 1 & 0 & 0 \\ 0 & 0 & 2 & 0 \\ 0 & 0 & 0 & 3 \\ 0 & 0 & 0 & 0 \bmr, \c M(S) = \bml 1 & 2 & 6 & 0\\ 0 & 1 & 4 & 18\\ 0 & 0 & 1 & 6 \\ 0 & 0 & 0 & 1 \bmr\]
\end{example}
\begin{question}
    What is the best matrix representation for an operator?
\end{question}
\begin{question}
    What information about eigenvalues/eigenvectors can be read off from a matrix representation?
\end{question}
\begin{theorem}
    Suppose $T \in \c L(V)$ and $\li{\vec v}n$ is a basis of $V$. Then the following are logically equivalent:
    \begin{enumerate}
        \item  $\c M(T)$ is upper triangular.
        \item $T\vec v_j \in \spa (\li{\vec v}j)$ $\forall j = 1,2, \ldots, n$.
        \item $\spa (\li{\vec v}j)$ is invariant under $T$ $\forall j = 1,2,\ldots, n$.
    \end{enumerate}
\end{theorem}
\newpage
\begin{proof}
    $1) \implies 2) $ \[\bml * & * & * & * & \cdots & * \\  & * & * & * & \cdots & * \\  &  & * & * & \cdots & * \\ & & & * & \cdots & * \\  & & & & \ddots & \vdots \\  & & & & & * \bmr\] We can see that $2)$ holds true by inspection. \\
    $2) \implies 3)$ Consider $T\vec v_h$ for $h \leq j$, by $2)$ we have $T\vec v_k \in \spa(\li{\vec v}h) \subseteq \spa (\li{\vec v}j)$. So $\spa (\li{\vec v}j)$ is invariant under $T$. \\
    $3) \implies 2)$ Consider $T\vec v_j$, by $3)$ it is a linear combination of $\li{\vec v}j$ because $T\vec v_j \in \spa(\li{\vec v}j)$ so $\c M(T)(i,j) = 0$ if $i > j$.
\end{proof}
\begin{question}
    What about conditions for lower-triangular matrices?
\end{question}
\begin{lemma}
Over $\b C$, every linear operator has at least one eigenvalue.
\end{lemma}
\begin{proof}
    Take $\vec v \in V\ \backslash \lb 0 \rb$, and consider the list $\vec v, T\vec v, T^2\vec v, \ldots, T^n\vec v$ where $n = \dim V$. There is a nontrivial linear combination of these vectors which is $0$. Suppose the equation \[a_0\vec v_1 + a_1T\vec v + a_2T^2\vec v + \cdots + a_nT^n\vec v = 0\]
    i.e. $p(T)v = 0$ for nonconstant $p(x) : = a_0 + a_1x + a_2x^2 + \cdots + a_nx^n$. By the fundamental theorem of algebra $p$ splits into linear factors over $\b C$.
    \[ p(x) = c(x - \lambda_1)(x - \lambda_2) \cdots (x - \lambda_m)\] for some $m \leq n$. Therefore 
    \[ p(T)v = c(T - \lambda_1 \b I)(T - \lambda_2  \b I) \cdots (T - \lambda_m \b I)\]
    Therefore at least one of these factors is not injective. This shows that $T$ has at least $1$ eigenvalue.
\end{proof}
\begin{theorem}
    For any $T \in \c L(V)$, $V$ is finite dimensional vector space over $\b C$, there exists its matrix representation $\c M(T)$ which is upper-triangular.
\end{theorem}
\newpage
\begin{proof} 
    We can induct on the dimension of $V$. 
    \textit{Base Step.} $n = 1$ is trivially true. \\
    \textit{Inductive Hypothesis.} Suppose Theorem holds for all vector spaces of dimension less than $\dim V$. \\
    \textit{Inductive Step.} Consider $\lambda \in \b C$ an eigenvalue of $T$ by lemma. We can define 
    \[U := \range (T  -\lambda \b I)\] 
    $U$ is a subspace of $V$. By the characterization of eigenvalues, $T - \lambda \b I$ is not surjective, hence $\range T - \lambda \b I \not\subseteq V$, hence $\dim \range (T - \lambda \b I) < \dim V$.
    We want to show that $U$ is invariant under $T$. Suppose $\vec v \in U$, then \[T\vec v = \underbrace{(T - \lambda \b I)\vec v}_{\in U} + \underbrace{\lambda \vec v}_{\in U}\] therefore we know that $U$ is invariant under $T$. 
    Consider \[T|_U \in \c L(U) : (T|_U)(\vec v) := T\vec  v \forall \vec v \in U\] 
    If $U \neq \lb 0 \rb$, then there is a basis $\li{\vec u}m$ of $U$ ($m < n$) such that the matrix representation of $T/U$ with respect to $\li{\vec u}m$ is upper triangular by the inductive hypothesis. Extend $\li{\vec u}m$ to a basis of $V$, $\li{\vec u}m, \li{\vec v}k$. We compute

    \[T\vec v_j = \underbrace{(T - \lambda \b I)\vec v_j}_{\in U = \spa (\li{\vec u}m)} + \lambda \vec v_j\] We also know that $T\vec u_l \in \spa (\li {\vec u}{l-1})$. We can see the matrix representation and hence we are done
    \[ \begin{array}{cc}
         \\ \\ \\ \\ m \\ \\ \\ 
    \end{array}\left[\begin{array}{cccc|ccccccc}
         * & * & \cdots & * & * & * & * \\
         0 & * & \cdots & * & * & * & * \\
         \vdots & \vdots & \ddots & \vdots & \vdots & \vdots & \vdots \\
         0 & 0 & \cdots & 0 & * & * & * \\
         \hline
         0 & 0 & \cdots & 0 & \lambda & 0 & 0 \\
         0 & 0 & \cdots & 0 & 0 & \lambda & 0 \\
         0 & 0 & \cdots & 0 & 0 & 0 & \lambda \\
    \end{array}\right]\]
\end{proof}
\begin{question}
    What about eigenvalues of a upper-triangular matrix?
\end{question}
\begin{theorem}
    An upper triangular matrix is invertible if and only if all its diagonal entries are nonzero.
\end{theorem}
\begin{proof}
    Suppose all diagonal entries are nonzero. Prove surjectivity. 
    \begin{align*}
        T\vec v_1  &=A_{1,1} v_1, A_{1,1} \neq 0 \implies \vec v_1 \in \range T\\
     T\vec v_2  &=A_{1,2} \vec v_1 + A_{2,2} \vec v_2 , A_{2,2} \neq 0 \implies \vec v_2 \in \range T \\
     \vdots & \hspace{5cm} \implies \\
     T\vec v_n &= A_{1,n} \vec v_1 + A_{2,n} \vec v_2 + \cdots + A_{n,n} \vec v_n \neq 0 \implies \vec v_n \in \range T
    \end{align*} 
    Therefore $\range T = V$, so $T$ is surjective, hence $T$ is invertible. 
    Suppose at least one one diagonal entry is $0$ we want to show that $T$ is not invertible. Say $A_{j,j} = 0$ for some $j$ and upper triangular matrix $A$. If $j = 1$, then $v_1 \in \null T$, hence $T$ is not invertible, and we are done. If $j > 1$, consider $U := \spa (\li{\vec v}j)$. $T$ maps $U$ to $\spa (\li{\vec v}{j-1})$. This shows $T|_U$ us not surjective, then we know that $T|_U$ is not injective and there exists $\vec u \in U$ such that $\vec u \in \null T|_U \implies \vec u \in \null T$. Therefore $T$ is not injective. Hence $T$ is not invertible.
\end{proof}
\begin{corollary}
    An upper triangular matrix / operator in upper triangular form has the diagonal elements / entries as its eigenvalues.
\end{corollary}
\begin{example}The matrix
    \[ A = \bml 
5 & * & * & * & * & \\
0 & 9 & * & * & * & \\
0 & 0 & 1 & * & * & \\
0 & 0 & 0 & 8 & * & \\
0 & 0 & 0 & 0 & 10 & \\
\bmr\]
has eigenvalue $1,5,9,8,10$.
\end{example}
\begin{example}
    $T: \c P_n(\b R) \to \c P_n(\b R) : p \mapsto 3p'' - 5'p' + 7p$ has eigenvalues $3,-5,7$.
\end{example}
\subsection{Eigenspaces and Diagonal Matrices}
\begin{definition}
    Suppose $T \in \c L(V)$ and $\lambda \in \b F$. The eigenspace of $T$ correspoding to $\lambda$, denoted as $E(\lambda, T)$ is defined as 
    \[ E(\lambda, T) := \lb \vec v \in V : T\vec v = \lambda \vec v \rb = \nul (T - \lambda I)\]
\end{definition}
\begin{definition}
    An operator $T \in \c L(T)$ is called diagonalizable if the operator has a diagonal matrix with repsect to some basis of $V$.
\end{definition}
\begin{theorem}
    For $T \in \c L(V)$, where $V$ is a finite dimensional vector space, then the following are equivalent
    \begin{enumerate}
        \item $\c M(T)$ is a diagonal matrix.
        \item the corresponding basis for $V$ consists of eigenvalue of $T$.
        \item $V = U_1 \oplus U_2 \oplus \cdots \oplus U_n$ where $\dim U_j = 1$ and $U_j$ is invariant under $T$ for all $j$.
        \item $V = W_1 \oplus W_2 \oplus \cdots \oplus W_k$, where $V/W_l = \lambda_l \b I$ for all $l$ and $W_l$ is invariant under $T$.
        \item $\dim V = \dim W_1 + \dim W_2 + \cdots + W_k$, where $W_e = \null (T - \lambda_e \b I)$.
    \end{enumerate}
\end{theorem}
\begin{proof}
Refer to Axler Page 157.
\end{proof}








